\documentclass{article}
\usepackage{amsmath, enumitem}

\begin{document}

\title{Econ 741 Homework 2}
\author{John Appert, Randall Chicolla, Andrea Franz}
\maketitle

\section{Chapter 4}

\subsection{Question 2}
Perform the following Operations on the following matrices. If it is impossible to perform the operations, explain why. Define the following matrices. \\

$A = \left( \begin{smallmatrix} 2&1&5\\ 1&2&5 \end{smallmatrix} \right)$
$B = \left( \begin{smallmatrix} 1&0\\0&1 \end{smallmatrix} \right)$
$C = \left( \begin{smallmatrix} 2&2\\1&1 \end{smallmatrix} \right)$
$D = \left( \begin{smallmatrix} 2&2&3\\1&1&5\\1&4&5 \end{smallmatrix} \right)$
$E = \left( \begin{smallmatrix} 6&3&7\\3&4&8\\3&7&5 \end{smallmatrix} \right)$\\

Part A. What is $B^{-1}$ \\
	
When taking the inverse of B by hand, we fist take the determinant to check if we can take the inverse to begin with. $|B|=(1*1-(0*0)=1$ Since the determinant is nonzero we take one over the determinant $1/1=1$ We then multiply by 1 and get the same matrix since the sign application and transpose that would normally be conducted in calculating the inverse results in no change in the matrix itself.\\

$B^{-1}=\left( \begin{smallmatrix} 1&0\\0&1 \end{smallmatrix} \right)$\\

If we confirm the result in stata we could enter the following into the do-file editor.\\

matrix C Equalsign (2,2forwardslash1,1)
matrix list C
matrix Binvsym Equalsign invsym(B)
matrix list Binvsym\\

Part B. What is $C^{-1}$?  \\

When calculating the determinant to check whether C is invertable, we find $|C|= (2*1)-(2*1)=0$, so there is no inverse which we can see by looking at the rows an recognizing that row two is a multiple of row one, so there would be linear dependence.\\

If we were to try to run is through STATA, we would get:

*matrix Cinvsym= invsym(C)\\
*matrix list Cinvsym\\
*Attempting to run Cinvsym results in "Matrix not symmetric" error \\

Part C. What is $A'$?\\

If taking the transpose of A, we merely turn our rows into columns and columns into rows.\\

This yields: $A' = \left( \begin{smallmatrix} 2&1\\1&2\\5&5 \end{smallmatrix} \right)$\\

If using STATA, we would enter:\\

matrix ATrans = A'\\
matrix list ATrans

Part D. What is $B'$\\

Similarly, we switch rows and columns, however in this case, since B is 2x2 identity matrix, its transpose is itself.\\

$B' = \left( \begin{smallmatrix} 1&0\\0&1 \end{smallmatrix} \right)$\\

If we were entering into STATA, it could look like the below:\\

matrix BTrans = B'\\
matrix list BTrans\\

Part E. What is $C'$?\\

When we switch the rows and columns, we get the below matrix as a result.\\

$C' = \left( \begin{smallmatrix} 2&1\\2&1 \end{smallmatrix} \right)$

If we enter into STATA, we could confirm the result.\\

matrix CTrans = C'\\
matrix list CTrans\\

Part F. What is AxC?\\

When looking at the dimensions of A, we see it is a 2x3. Similarly, C is a 2x2. Since the number of columns in A (3) do not match the number of rows in C (2), the matrix would be non-conformable unless we were to take the transpose of A first, which is done in part G.\\

In Stata, entering results in:\\
*Part F What is A x C ?\\
*matrix $A \times C$ = A * C\\
*matrix list $A\times C$\\
*Attempting to run $A\times C$ results in "conformability error" \\

\section{Chapter 5}

\subsection{R}

We decided to directly import the dta file into R in order to complete all the data munging and statistical analysis.  Standard R libraries are unable to read stata 13 files.  We installed the package "readstat13" which allowed us to directly import the file into R.
\\
After importing the data we converted it to a dataframe.  After converting to the dataframe we converted all columns to numeric data and removed any rows where the wages were equal to zero or greater than ninety-nine million.  
\\
The three componenets of the standard estimator are the following:
\\

\begin{equation}
\hat{\sigma^2}=\frac{\hat{\epsilon'} \hat{\epsilon}}{N-k}:
\end{equation}

We found the following values for these components of the standard error using R:
\\
\begin{equation}
var(\hat{\beta})=\sigma^2 \times (X'X)^{-1}
\end{equation}
\begin{equation}
\hat{\sigma^2}=3,211,034,129
\end{equation}
\begin{equation}
\hat{\epsilon'}\hat{\epsilon}=4.990559e+15
\end{equation}
\begin{equation}
N-k=1,527,729
\end{equation}
\begin{equation}
(X'X)^{-1}=
\begin{bmatrix}

3.981391e-09 & -1.565183e-07 \\
-1.565183e-07 & 6.807685e-06
\end{bmatrix}

\end{equation}

Based on these calculations we find that the standard error on the constant is 147.850 and the standard error on the X values is 3.576.

\end{document}
