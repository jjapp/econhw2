\documentclass{article}
\usepackage{amsmath, enumitem}

\begin{document}

\title{Econ 741 Homework 2}
\author{John Appert Randall Chicolla Andrea Franz}
\maketitle

\section{Chapter 4}

\subsection{Question 1}

List the math classes we've taken in college or at the graduate level.

John Appert:

\begin{itemize}
\item Calculus 1
\item Calculus 2
\item Calculus 3 w/ Vector Fields
\item Differential Equations
\item Engineering Mathmatics
\item Math of Quantum Mechanics
\item Group Theory for Quantum Mechanics
\end{itemize}

Randy Chicolla:

\begin{itemize}

\end{itemize}

Andrea Franz:

\begin{itemize}

\end{itemize}

\subsection{Question 2}

Perform the following operations on the following matrices:

\begin{equation}
\begin{bmatrix}
A
\end{bmatrix}
=\begin{bmatrix}
2&1&5\\
1&2&5
\end{bmatrix}
\end{equation} 

\begin{equation}
\begin{bmatrix}
B
\end{bmatrix}
=
\begin{bmatrix}
1&0\\
0&1
\end{bmatrix}
\end{equation}

\begin{equation}
\begin{bmatrix}
c
\end{bmatrix}
=
\begin{bmatrix}
2&2\\
1&1
\end{bmatrix}
\end{equation}

\begin{equation}
\begin{bmatrix}
D
\end{bmatrix}
=
\begin{bmatrix}
2&2&3\\
1&1&5\\
1&4&5
\end{bmatrix}
\end{equation}

\begin{equation}
\begin{bmatrix}
E
\end{bmatrix}
=\begin{bmatrix}
6&3&7\\
3&4&8\\
3&7&5
\end{bmatrix}
\end{equation}
\begin{enumerate}[label=\alph*]
\item What is $B^{-1}$?
\item What is $C^{-1}$?
\item What is A'?
\item What is B'?
\item What is C'?
\item What is $A \times C$?
\item What is $A' \times C$?
\item What is $B \times C$?
\item What is $B \plus C$?
\item What is $D^{-1}$?
\item What is $D \times E$?

\end{enumerate}

\section{Chapter 5}

\subsection{R}

We decided to directly import the dta file into R in order to complete all the data munging and statistical analysis.  Standard R libraries are unable to read stata 13 files.  We installed the package "readstat13" which allowed us to directly import the file into R.
\\
After importing the data we converted it to a dataframe.  After converting to the dataframe we converted all columns to numeric data and removed any rows where the wages were equal to zero or greater than ninety-nine million.  
\\
The three componenets of the standard estimator are the following:
\\

\begin{equation}
\hat{\sigma^2}=\frac{\hat{\epsilon'} \hat{\epsilon}}{N-k}:
\end{equation}

We found the following values for these components of the standard error using R:
\\
\begin{equation}
\hat{\sigma^2}=3,211,034,129
\end{equation}
\begin{equation}
\hat{\epsilon'}\hat{\epsilon}=4.990559e+15
\end{equation}
\begin{equation}
N-k=1,527,729
\end{equation}
\begin{equation}
(X'X)^{-1}=
\begin{bmatrix}

3.981391e-09 & -1.565183e-07 \\
-1.565183e-07 & 6.807685e-06
\end{bmatrix}

\end{equation}

Based on these calculations we find that the standard error on the constant is 147.850 and the standard error on the X values is 3.576.

\end{document}
